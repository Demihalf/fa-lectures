\section{Компактные операторы}
Далее $X$ и $Y$ --- банаховы пространства.

\begin{definition}
    Оператор $A \in L(X, Y)$ называется \emph{компактным}, если образ $A(M)$ всякого ограниченного
    множества $M \subset X$ есть предкомпактное множество в $Y$.

    Множество компактных операторов, действующих из $X$ в $Y$ будем обозначать
     $\comp{X,Y}$. Как обычно, если $X = Y$, пишут $\comp{X}$.
\end{definition}

Можно показать, что для компактности оператора $A$ достаточно показать предкомпактность образа
единичного шара.

\begin{remark}
Из теоремы Хаусдорфа следует, что оператор компактен тогда и только тогда, когда для каждой
ограниченной последовательности $\menge{x_n}$ из последовательности $\menge{Ax_n}$ можно выжелить
сходящуюся в $Y$ подпоследовательность.
\end{remark}

\begin{definition}
    Оператор $A \in L(X, Y)$ называется \emph{оператором с конечным рангом}, если его образ
    $\im A$ есть конечномерное подпространство в $Y$.
\end{definition}

\begin{theorem}
    Для того чтобы множество из конечномерного банахова пространства было предкомпактно,
    необходимо и достаточно чтобы оно было ограничено. Как следствие, такое множество компактно
    тогда и только тогда, когда оно замкнуто и ограничено.
\end{theorem}

\begin{theorem}
    Всякий оператор с конечным рангом компактен.
\end{theorem}

\begin{proof}
    Поскольку $A$ ограничен, он переводит ограниченное множество $M$ в ограниченное. Но поскольку
    $\im A$ конечномерен, то по предыдущей теореме $A(M)$ предкомпактно.
\end{proof}

\begin{example}
    Пусть $A \colon C[a, b] \to C[a, b]$ --- интегральный оператор с ядром 
    $K \in C([a,b] \times [a,b])$.

    Используя теорему Арцела, можно показать, что всякий интегральный оператор компактен.

    Ядро $K$ называется \emph{вырожденным}, если его можно представить в виде
    \[ K(t, s) = \sum_{i=1}^n p_i(t) q_i(s), \]
    где $p_i, q_i \in C[a, b]$ и $p_i$ линейно независимы.

    Оператор с вырожденным ядром является оператором с конечным рангом. В самом деле:
    \[ (Ax)(t) = \sum_{i=1}^n \, p_i(t) \!\! \int_a^b \! q_i(s) x(s) \dd s, \]
    то есть всякая функция $Ax \in C[a,b]$ представима в виде линейной комбинации 
    $p_i$, которые линейно независимы, а значит образуют базис в $\im A$.
\end{example}

\begin{definition}
    Подмножество $I \subset A$ алгебры $A$ называется \emph{идеалом (двусторонним идеалом)}, если
    оно является подпространством в $A$ и для всех $a \in A$ и $b \in I$ справедливы равенства
    \[ ab \in I, \quad ba \in I. \]
\end{definition}

\begin{theorem}
    Множество $\comp{X,Y}$ образует замкнутое подпространство в $L(X, Y)$. Если $X = Y$,
    то $\comp{X}$ --- двусторонний идеал в банаховой алгебре $L(X)$.
\end{theorem}

\begin{proof}
    Докажем, что $\comp{X,Y}$ --- подпространство в $L(X, Y)$. Пусть $A, B \in \comp{X, Y}$. Если
    $\menge{x_n} \subset X$ --- ограниченная последовательность, то из
    $\menge{(\alpha A + \beta B)x_n}$ можно выделить сходящуюся,
    выделив сходящуюся сначала из последовательности $\menge{Ax_n}$ --- $\menge{Ax_{n_k}}$, а 
    затем выделить сходящуюся из $\menge{Bx_{n_k}}$ --- $\menge{Bx_{n_{k_i}}}$. Тогда 
    последовательность
    $\menge{(\alpha A + \beta B)x_{n_{k_i}}}$ также будет сходящейся, то есть линейная комбинация
    компактных операторов также является компактным оператором.

    Покажем, что $\comp{X,Y}$ замкнуто в $L(X,Y)$. Пусть $\menge{A_n} \subset \comp{X, Y}$ сходится
    по норме к $A$, то есть $\norm{A_n - A} \to 0$. Покажем, что $A$ компактен. Для этого покажем,
    что образ единичного шара $B(0,1)$ вполне ограничен (тогда, по теореме Хаусдорфа, он
    предкомпактен), то есть нужно доказать, что для каждого $\varepsilon > 0$ множество $A(B(0,1))$
    можно покрыть конечным числом шаров радиуса $\varepsilon$.

    Зафиксируем $\varepsilon > 0$. Пусть $m$ таково, что $\norm{A_m - A} < \varepsilon/2$.
    Поскольку $A_m(B(0,1))$ вполне ограниченное множество, по $\varepsilon/2$ для него найдется 
    конечное покрытие шарами радиуса $\epsilon/2$ с центрами в точках $y_i$, $i = \overline{1,k}$:
    \[ A_m(B(0,1)) \subset \bigcup_{i=1}^k B(y_i, \frac{\varepsilon}{2}). \]
    Покажем, что 
    \[ A(B(0,1)) \subset \bigcup_{i=1}^k B(y_i, \varepsilon). \]
    В самом деле, пусть $x \in B(0, 1)$ и $A_mx \in B(y_i, \varepsilon/2)$. Тогда
    \[ \norm{A_mx - Ax} < \frac{\varepsilon}{2} \]
    и
    \[ \norm{Ax - y_i} \leq \norm{Ax - A_mx} + \norm{A_mx - y_i} < \varepsilon, \]
    то есть $Ax$ лежит в шаре $B(y_i, \varepsilon) \subset \bigcup_{i=1}^k B(y_i, \varepsilon)$.
    Компактность оператора $A$ доказана, то есть $\comp{X,Y}$ --- замкнутое подпространство.

    Осталось доказать, что $\comp{X}$ образует двусторонний идеал в $L(X)$. Пусть $A \in \comp{X}$,
    $B \in L(X)$. Нужно показать, что $AB, BA \in \comp{X}$. Пусть $\menge{x_n}$ --- ограниченная
    последовательность в $X$. $\menge{Bx_n}$ также ограничена. Поскольку оператор $A$ компактен,
    из последовательности $\menge{A(Bx_n)}$ можно выделить сходящуюся, что в точности и означает,
    что $AB \in \comp{X}$. Из последовательности $\menge{Ax_n}$ также можно выделить сходящуюся
    $\menge{Ax_{n_k}}$, но тогда и $\menge{B(Ax_{n_k})}$ сходится, значит $BA \in \comp{X}$.
\end{proof}

\begin{lemma}[о почти перпендикуляре]
    Пусть $X$ --- банахово пространство, $M \subset X$ --- замкнутое подпространство,
    не совпадающее со всем $X$. Тогда
    для любого $\varepsilon > 0$ найдется такой $x \in X \setminus M$, $\norm{x} = 1$, что
    \[ 1 - \inf_{m\in M} \norm{x - m} < \varepsilon. \]
\end{lemma}

\begin{theorem}
    Пусть $X$ --- бесконечномерное банахово пространство. Тогда замкнутый шар 
    $\overline{B(a, r)}$ не является компактом.
\end{theorem}

\begin{proof}
    Докажем утверждение для единичного шара (общее утверждение следует). Возьмем произвольный
    $x_0$, $\norm{x_0} = 1$. Определим подпространство $M_1 = \linspan\menge{x_0}$. 
    По лемме о почти перпендикуляре для $\varepsilon = 1/2$ найдется такой 
    $x_1 \in X \setminus M_1$, $\norm{x_1} = 1$, что
    $\norm{x_1 - x_0} > 1/2$. Для подространства $M_2 = \linspan\menge{x_0, x_1}$ также 
    справедлива лемма о почти перпендикуляре, значит найдется $x_3 \in X \setminus M_2$, 
    $\norm{x_3} = 1$, что $\norm{x_2 - x_0} > 1/2$ и $\norm{x_2 - x_1} > 1/2$. Продолжая аналогично, получим
    последовательность $\menge{x_k}$ единичных векторов, находящихся друг от друга на расстоянии 
    большем $1/2$.
    Очевидно, что из такой последовательности выделить сходящуюся нельзя, а значит множество 
    $\overline{B(0, 1)}$ не предкомпактно.
\end{proof}
