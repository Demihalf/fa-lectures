\section{Элементы теории функции комплексной переменной}
Более подробную информацию можно найти, например, в книге Шабата Б. В. <<Введение
в комплексный анализ>>.

Рассмотрим функцию $F \colon E \subset \fieldr^2 \to \fieldr^2$. Её можно представить в виде
\[ F(x, y) = (P(x, y), Q(x, y)), \]
где $P, Q \colon E \to \fieldr$. Более того, эту функцию можно рассматривать как
функцию $F \colon E \subset \fieldc \to \fieldc$:
\[ F(x + iy) = P(x, y) + i Q(x, y). \]
Пусть теперь $F$ дифференцируема в точке $x_0 \in E$ как отображение из $\fieldr^2$ в $\fieldr^2$.
Исследуем, при каких условиях эта функция будет дифференцируема как отображение из $\fieldc$ в
$\fieldc$. Заметим, что существуют функции, для которых это не выполняется. Примером может служить функция
\[f\colon \fieldc \to \fieldc, \quad f(z) = \Re z. \]
Если рассматривать эту функцию как отображение $f \colon \fieldr^2 \to \fieldr^2$, то легко видеть,
что это линейный оператор:
\[f(x_1, x_2) = x_1. \]
То есть $f$ дифференцируемо в каждой точке из $\fieldr^2$. Однако если мы рассмотрим предел
\begin{equation}\label{eq:diff}
 \lim_{z\to 0} \frac{f(z) - f(0)}{z} = \lim_{z\to 0} \frac{\Re z}{z}
\end{equation}
при стремлении $z$ к нулю вдоль мнимой оси и вдоль действительной оси
\[ \lim_{\substack{z\to 0\\ \Re z = 0}} \frac{\Re z}{z} = 0; \]
\[ \lim_{\substack{z\to 0\\ \Im z = 0}} \frac{\Re z}{z} = \frac{z}{z} = 1. \]

Таким образом предел \eqref{eq:diff} не существует, то есть $f \colon \fieldc \to \fieldc$ не
дифференцируема.

Как известно, если отображение $f \colon \fieldr^2 \to \fieldr^2$ дифференцируемо в точке 
$x_0 = (x, y)$,
то у него существуют частные производные первого порядка, и матрица Якоби в точке $x_0$ 
есть матрица оператора $f'(x_0)$ в стандартном базисе. 

Пусть $f(x, y) = (u(x, y), v(x, y))$.
\[ f'(x_0) \sim 
    \renewcommand{\arraystretch}{2}
    \begin{pmatrix}
    \dfrac{\partial u(x, y)}{\partial x} & \dfrac{\partial u(x, y)}{\partial y} \\
    \dfrac{\partial v(x, y)}{\partial x} & \dfrac{\partial v(x, y)}{\partial y} 
    \end{pmatrix}
\]

Следующее утверждение вытекает из представления комплексных чисел в виде матрицы и утверждения,
что все линейные операторы в $\fieldc$ действуют по правилу $x \mapsto \alpha x$, 
$\alpha \in \fieldc$.

\begin{lemma}
    Для того чтобы матрица $A = \begin{pmatrix}
    a & b \\
    c & d 
    \end{pmatrix}$ с вещественными коэффициентами задавала
    линейный оператор в комплексном линейном пространстве $\fieldc$, необходимо и достаточно, чтобы
    \[ \left\{\begin{array}{l}
        a = d \\
        b = -c
        \end{array}\right.\]
\end{lemma}

Непосредственно из леммы получаем
\begin{theorem}[условия Коши-Римана]
    Дифференцируемое \\ в точке $x_0 = (x, y) \in U \subset \fieldr^2$ 
    отображение $f \colon U \subset \fieldr^2 \to \fieldr^2$,
    $f(x, y) = (u(x, y), v(x, y))$
    дифференцируемо как отображение $U \subset \fieldc \to \fieldc$ в том и только в том случае, если выполняются следующие условия (условия Коши-Римана):
    \[ \left\{\begin{array}{l}
        \dfrac{\partial u(x, y)}{\partial x} 
            = \dfrac{\partial v(x, y)}{\partial y} \\
        \dfrac{\partial u(x, y)}{\partial y}
            = -\dfrac{\partial v(x, y)}{\partial x}
        \end{array}\right. \]
\end{theorem}

\begin{definition}
    Функция $f \colon U \subset \fieldc \to \fieldc$ называется \emph{аналитической} 
    (чаще говорят \emph{голоморфной}) на открытом множестве $U$, если она дифференцируема 
    (как функция в комплексном пространстве) 
    в каждой точке множества $U$. 
\end{definition}

\begin{definition}
    \emph{Путём} в $U \subset \fieldc$ называется непрерывное отображение
    $\gamma \colon [0, 1] \to U$. Если это отображение является кусочно непрерывно
    дифференцируемым, то его называют \emph{кусочно гладким путём}.
\end{definition}

\begin{definition}
    \emph{Интегралом} от комплексной функции \\$f \colon U \subset \fieldc \to \fieldc$ \emph{вдоль
        кусочно гладкого пути} $\gamma \colon [0, 1] \to U$ называется 
    \[ \int_\gamma f(z) \dd z := \int_0^1 f(\gamma(t)) \gamma'(t) \dd t. \]
\end{definition}

Если $f(x + i y) = u(x, y) + i v(x, y)$, то интеграл можно записать в виде
\begin{multline*}
 \int_\gamma f(z) \dd z = \int_\gamma (u(x, y) + i v(x, y)) \dd (x + i y) =\\ = 
   \int_\gamma u(x, y) \dd x - v(x, y) \dd y + i \int_\gamma v(x, y) \dd x + u(x, y) \dd y, 
\end{multline*}
где в правой части стоят известные из курса анализа криволинейные интегралы второго рода.

\begin{theorem}[Коши]
    Если $U$ --- односвязная область, граница которой есть простая замкнутая кривая, задаваемая
    кусочно гладким путём (контуром) $\gamma$,
    то для любой аналитической функции $f \colon U \subset \fieldc \to \fieldc$
    справедливо равенство
    \[ \int_\gamma f(z) \dd z = 0. \]
\end{theorem}

\begin{proof}
    Вспомним известную из анализа формулу Грина:
    \[ \int_\gamma P \dd x + Q \dd y = \iint_U \left(\dfrac{\partial Q}{\partial x} - \dfrac{\partial P}{\partial y}\right) \dd x \dd y. \]
    Тогда, применяя формулу Грина и условия Коши-Римана, получаем:
    \begin{multline*}
     \int_\gamma f(z) \dd z =  
       \int_\gamma u(x, y) \dd x - v(x, y) \dd y + i \int_\gamma v(x, y) \dd x + u(x, y) \dd y = \\ = \iint_U \left(-\dfrac{\partial v}{\partial x} - \dfrac{\partial u}{\partial y}\right) \dd x \dd y + i \iint_U \left(\dfrac{\partial u}{\partial x} - \dfrac{\partial v}{\partial y}\right) \dd x \dd y = 0.
    \end{multline*}
\end{proof}

\begin{theorem}[интегральная формула Коши]
    Пусть \\$f \colon U \subset \fieldc \to \fieldc$ --- аналитическая функция, определенная
    в области $U$. Тогда для всех $z \in U$
    \[ f(z) = \frac{1}{2 \pi i} \int_\gamma \frac{f(\lambda)}{\lambda - z} \dd \lambda, \]
    где $\gamma$ --- контур, окружающий точку $z$, причем обход контура ведется против часовой
    стрелки.
\end{theorem}
