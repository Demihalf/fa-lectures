\section{Гильбертовы пространства}
\subsection{Начальные сведения}
\begin{definition}
    Линейное пространство $H$ над полем $\fieldk$ называется 
    \emph{пространством со скалярным произведением},
    если в нем задана функция $\langle \cdot, \cdot \rangle \colon H \times H
    \to \fieldk$, такая что для всех $x, y, z \in H$ и $\alpha, \beta \in \fieldk$
    справедливы следующие свойства:
    \begin{enumerate}
        \item $ \langle x, x \rangle = 0 \Leftrightarrow x = 0 $
            (невырожденность);
        \item $ \langle x, x \rangle \geq 0 $ (положительная определённость);
        \item $ \langle \alpha x + \beta y, z \rangle = \alpha \langle x, z
                \rangle + \beta \langle y, z \rangle $ (линейность по первому
                аргументу);
        \item $ \langle x, y \rangle = \overline{\langle y, x \rangle} $ (эрмитова
            симметричность).
    \end{enumerate}
    Такая функция называется \emph{скалярным произведением}.
\end{definition}

Далее будем рассматривать только комплексные пространства со скалярным
произведением.

В пространстве со скалярным произведением можно ввести норму по формуле
\begin{equation}\label{eq:hilbert_norm}
    \norm{x} = \sqrt{\langle x, x \rangle}. 
\end{equation}

Неравенство треугольника следует из неравенства
\begin{equation}\label{eq:schwarz}
    \absv{\langle x, y \rangle} \leq \norm{x} \norm{y},
\end{equation}
которое называют \emph{неравенством Коши-Буняковского-Шварца} или просто
неравенством Шварца.

\begin{theorem}\label{th:cont}
    Пусть $\menge{x_n}, \menge{y_n}$ --- последовательности из $H$,
    причем $x_n \to x$, $y_n \to y$. Тогда $\langle x_n, y_n \rangle \to \langle
    x, y \rangle$.
\end{theorem}

\begin{proof}
    Используем неравенство Шварца:
    \begin{multline*}
        \absv{\langle x_n , y_n \rangle - \langle x, y \rangle} = \absv{\langle
            x_n, y_n \rangle - \langle x_n, y \rangle + \langle x_n, y \rangle -
        \langle x, y \rangle} \leq \\ \leq \absv{\langle x_n, y_n - y \rangle} +
        \absv{\langle x_n - x, y \rangle} \leq \\ \leq \norm{x_n}\norm{y_n - y} +
        \norm{x_n - x}\norm{y} \to 0, \quad n \to \infty
    \end{multline*}
\end{proof}

\begin{definition}
    Если пространство со скалярным произведением полно по норме, определённой
    равенством \eqref{eq:hilbert_norm}, то оно
    называется \emph{гильбертовым пространством}.
\end{definition}

\begin{example}
    Лебегово пространство $L^2(E, \mu)$ является гильбертовым пространством 
    со скалярным произведением, определённым по формуле
    \[ \langle f, g \rangle = \int_E f(x) \overline{g(x)} \dd \mu(x). \]
    Существование этого интеграла следует из неравенства
    \[ \absv{f(x)\overline{g(x)}} \leq \frac{\absv{f(x)}^2 + \absv{g(x)}^2}{2}.
    \]
\end{example}

\begin{example}
    В частности, гильбертовым пространством является пространство суммируемых с
    квадратом последовательностей $\ell^2$. Скалярное произведение задаётся
    формулой
    \[ \langle x, y \rangle = \sum_{k=1}^\infty x_n \overline{y_n}. \]
    Сходимоcть ряда обеспечивается аналогичной оценкой.
\end{example}

\begin{definition}
    Векторы $x, y \in H$ называются \emph{ортогональными}, если $\langle x, y \rangle =
    0$. При этом пишут $x \perp y$.
\end{definition}

\begin{definition}
    Пусть $M \subset H$ --- множество из $H$. Тогда говорят, что вектор
    $x \in H$ \emph{ортогонален} $M$, если $x$ ортогонален любому вектору $m \in
    M$ (в этом случае используется обозначение $x \perp M$).
\end{definition}

\begin{theorem}
    Для всех векторов $x, y \in H$ выполняется тождество параллелограмма:
    \[ \norm{x + y}^2 + \norm{x - y}^2 = 2\norm{x}^2 + 2\norm{y}^2. \]
\end{theorem}

\begin{proof}
    Доказывается элементарными преобразованиями.
\end{proof}

\subsection{Теорема об ортогональном дополнении}
\begin{definition}
    Множество $A \subset X$ называется \emph{выпуклым}, если для любых векторов $a, b
    \in A$ векторы вида $(1-t) a + t b, t \in [0, 1]$ также лежат в $A$.
\end{definition}

Очевидно, всякое подпространство в нормированном пространстве является выпуклым
множеством.

\begin{theorem}[о наилучшем приближении]
    Пусть $A$ --- непустое выпуклое замкнутое множество в гильбертовом пространстве $H$.
    Тогда для любого $x \in H \setminus A$ найдётся единственный вектор $a_0 \in
    A$ такой, что
    \[ \norm{x - a_0} = \inf_{a \in A} \norm{x - a}. \]
    Иначе говоря, в $A$ найдется вектор $a_0$, который находится от $x$ на наименьшем
    возможном расстоянии. Такой вектор $a_0$ называется элементом наилучшего
    приближения вектора $x$ в множестве $A$.
\end{theorem}

\begin{proof}
    По определению нижней грани, существует такая последовательность $\menge{a_n}$
    элементов из $A$, что 
    \[d_n = \norm{x - a_n} \to \inf_{a\in A} \norm{x - a} = d.\]
    Покажем, что эта последовательность фундаментальна.

    По тождеству параллелограмма получаем:
    \begin{multline*}
        \norm{(a_n - x) + (a_m - x)}^2 + \norm{(a_n-x)-(a_m-x)}^2 = \\ =
        2\left(\norm{a_n-x}^2+\norm{a_m-x}^2\right). 
    \end{multline*}

    Заметим, что правая часть равенства стремится к $4d^2$ при стремлении $n$ и
    $m$ к бесконечности. Разделим обе части равенства на $4$:
    \begin{multline*}
        \frac{1}{4}\left(\norm{(a_n - x) + (a_m - x)}^2 +
        \norm{(a_n-x)-(a_m-x)}^2\right) = \\ =
        \frac{1}{2}\left(\norm{a_n-x}^2+\norm{a_m-x}^2\right). 
    \end{multline*}

    После преобразований получаем:
    \[
        \norm{\frac{a_n+a_m}{2}-x}^2+\frac{\norm{a_n-a_m}^2}{4} =
        \frac{1}{2}\left(\norm{a_n-x}^2+\norm{a_m-x}^2\right).
    \]

    Поскольку множество $A$ выпукло, вектор $(a_n+a_m)/2$ принадлежит
    $A$, а значит, в силу определения нижней грани, справедлива оценка
    \[ \norm{\frac{a_n+a_m}{2}-x}^2 \geq d^2. \]

    Тогда
    \begin{multline*}
        \frac{\norm{a_n-a_m}^2}{4} =
        \frac{1}{2}\left(\norm{a_n-x}^2+\norm{a_m-x}^2\right) -
        \norm{\frac{a_n+a_m}{2}-x}^2 \leq \\ \leq
        \frac{1}{2}\left(\norm{a_n-x}^2+\norm{a_m-x}^2\right) - d^2.
    \end{multline*}

    Правая часть неравенства стремится к нулю, а это значит, что
    $\norm{a_n-a_m}$ также стремится к нулю, что и означает фундаментальность
    последовательности $\menge{a_n}$.

    Поскольку пространство полно, существует вектор $a_0 =
    \lim\limits_{n\to \infty} a_n$. В силу замкнутости множества $A$ этот вектор также
    лежит в $A$. При этом
    \[ \norm{x-a_0} \leq \norm{x-a_n} + \norm{a_n-a_0} \to d, \quad n \to
    \infty, \]
    то есть $\norm{x - a_0} = d$, что и означает, что $a_0$ является элементом
    наилучшего приближения $x$ в $A$.

    Покажем, что других векторов наилучшего приближения в $A$ нет. Пусть $a_0'
    \in A$ и $\norm{a_0' - x} = d$. Тогда, снова используя тождество
    параллелограмма, получаем
    \[ 4 \norm{x - \frac{a_0-a_0'}{2}}^2 + \norm{a_0 - a_0'}^2 = 2 \norm{x -
    a_0}^2 + 2 \norm{x - a_0'}^2 = 4d^2.\]

    Первый квадрат нормы не меньше $4d^2$, откуда следует, что второй не
    превосходит нуля, а значит
    \[ \norm{a_0-a_0'}^2 = 0, \]
    то есть $a_0 = a_0'$.
\end{proof}

\begin{definition}
    Пусть $M \subset H$ --- подпространство из $H$. Вектор $a \in M$ называется
    \emph{проекцией} вектора $x \in H$ на $M$ если $x - a \perp M$, то есть для всех $m
    \in M$ выполняется равенство
    \[ \langle x - a, m \rangle = 0. \]
\end{definition}

\begin{theorem}\label{th:projection}
    Если $M \subset H$ --- замкнутое подпространство, $x \in H \setminus M$,
    то тогда вектор $a \in M$ является проекцией $x$ на $M$ тогда и только тогда, 
    когда $a$ --- элемент наилучшего приближения $x$ в $M$.
\end{theorem}

\begin{proofbreak}
    \dindent \textbf{Необходимость:}

    Пусть $x - a \perp M$. Тогда по теореме Пифагора для любого $m \in M$
    справедливо равенство
    \[ \norm{x - m}^2 = \norm{x - a}^2 + \norm{a - m}^2. \]
    Значит,
    \[ \inf_{m \in M} \norm{x - m} = \norm{x - a},\]
    откуда и следует, что $a$ --- элемент наилучшего приближения $x$ в $M$.

    \textbf{Достаточность:}

    Пусть $a \in M$ --- элемент наилучшего приближения $x$ в $M$, то есть
    \[ \inf_{m \in M} \norm{x - m} = \norm{x - a} = d.\]
    Покажем, что для любого $m \in M$ выполнено равенство $\langle x - a, m
    \rangle = 0$.

    Обозначим $x - a = z$ и пусть $t \in \fieldr$. Тогда
    \begin{multline*}
        \norm{x - (a + tm)}^2 = \norm{z - tm}^2 = \langle z - tm, z - tm \rangle
        = \\ = \norm{z}^2 - 2 t \Re \langle z, m \rangle + t^2 \norm{m}^2 = d^2
        - 2 t \Re \langle z, m \rangle + t^2 \norm{m}^2.
    \end{multline*}

    Поскольку $a + tm \in M$, $\norm{x - (a+tm)}^2 \geq d^2$, откуда
    \[ d^2 - 2 t \Re \langle z, m \rangle + t^2 \norm{m}^2 \geq d^2, \]
    то есть при всех $t \in \fieldr$
    \[ t^2 \norm{m}^2 - 2 t \Re \langle z, m \rangle \geq 0, \]
    что возможно только в случае $\Re \langle z, m \rangle = 0$. 
    
    Взяв теперь вместо $t$ величину $it$, $t \in \fieldr$, можно аналогично
    показать, что $\Im \langle z, m \rangle = 0$, что в совокупности даёт
    \[ \langle z, m \rangle = 0, \]
    то есть $x - a \perp M$.
\end{proofbreak}

Таким образом мы доказали, что для всякого замкнутого подпространства $M \in H$
и вектора $x \in H\setminus M$ существует проекция $x$ на $M$, причём она совпадает с
элементом наилучшего приближения $x$ в $M$.

\begin{definition}
    \emph{Ортогональным дополнением} множества $A$ в гильбертовом пространстве $H$ называется множество
    \[ A^\perp = \menge{x \in H : x \perp A}. \]
\end{definition}

Из свойств скалярного произведения и теоремы \ref{th:cont} нетрудно видеть, 
что $A^\perp$ --- замкнутое подпространство из $H$ для любого
подмножества $A \hm\subset H$.

\begin{theorem}[об ортогональном дополнении]
    Если $M$ --- замкнутое подпространство из $H$, то $H = M \oplus M^\perp$.
\end{theorem}

\begin{proof}
    Покажем, что всякий вектор $x \in H$ можно представить в виде суммы векторов
    из $M$ и $M^\perp$. Пусть $a$ --- элемент наилучшего приближения $x$ в $M$.
    Тогда по предыдущей теореме $x - a \perp M$, то есть $x - a \in M^\perp$,
    откуда получаем
    \[ x = a + (x - a), \]
    где $a \in M$, $x - a \in M^\perp$.

    Единственность такого представления обеспечивается тем фактом, что
    \[ M \cap M^\perp = \menge{0}. \]
\end{proof}

\subsection{Базис в гильбертовом пространстве}

\begin{definition}
    Банахово пространство $X$ называется \emph{сепарабельным}, если существует такое
    счетное множество $M \subset X$, что $\overline{M} = X$, то есть, как еще говорят, $M$
    всюду плотно в $X$.
\end{definition}

\begin{definition}
    Множество $M \subset H$ называется \emph{ортонормированным}, если для всех $x, y \in M$
    \begin{enumerate}
        \item $ \norm{x} = 1; $
        \item $ x \neq y \Leftrightarrow \langle x, y \rangle = 0. $
    \end{enumerate}
\end{definition}

\begin{lemma}     
В сепарабельном гильбертовом пространстве всякое ортонормированное множество не
более чем счетно.
\end{lemma}

\begin{proof}
    Пусть $E$ --- ортонормированное множество в сепарабельном гильбертовом пространстве $H$.
    Тогда для любых векторов $e_1$, $e_2$ из $E$ справедливо (проверяется непосредственно):
    \[ \norm{e_1 - e_2} = \sqrt{2}. \]

    Поскольку $H$ сепарабельно, существует счетное множество $F = \menge{f_k}$,
    такое что для любого $e \in E$ найдется $f \in F$, что $\norm{e - f} < \sqrt{2}/2$.
    Но тогда, если $\norm{e_1 - f} < \sqrt{2}/2$, то (из неравенства треугольника)
    \[ \norm{e_2 - f} \geq \norm{e_1-e_2} - \norm{e_1 - f} > \frac{\sqrt{2}}{2}, \]
    то есть двум разным $e_1$ и $e_2$ не может соответствовать один и тот же $f$ с вышеуказанным свойством, то есть существует инъективное отображение $E$ в $F$.
    Из этого следует, что множество $F$ имеет мощность, не меньшую чем множество $E$, то есть $E$ --- не более, чем счетно.
\end{proof}

\begin{lemma}\label{le:series}
Пусть $\menge{e_n}$ --- ортогональная последовательность векторов из $H$.
Тогда следующие условия эквивалентны:
\begin{enumerate}
    \item $\sum\limits_{n=1}^\infty e_n$ сходится;
    \item $\sum\limits_{n=1}^\infty \norm{e_n}^2$ сходится
\end{enumerate}
\end{lemma}

\begin{proof}
    По теореме Пифагора
    \[ \norm{\sum_{k = m+1}^n e_k}^2 = \sum_{k=m+1}^n \norm{e_k}^2. \]
    Из этого равенства и полноты пространства утверждение теоремы следует немедленно.
\end{proof}

Далее $H$ --- сепарабельное гильбертово пространство.
\begin{definition}
    Последовательность $\menge{e_n}$ называется \emph{ортонормированным базисом} (Шаудера) в $H$, если выполнены следующие условия:
    \begin{enumerate}
        \item Элементы последовательности $\menge{e_n}$ образуют ортонормированное множество;
        \item Если $a \perp e_k$ для всех $k \in \mathbb N$, то $a = 0$ (свойство полноты).
    \end{enumerate}
\end{definition}

\begin{definition}
    Пусть $\menge{e_n}$ --- ортонормированный базис в~$H$. Тогда \emph{рядом Фурье} вектора $x \in H$ называется ряд 
    \[ \sum\limits_{k=1}^\infty \langle x, e_k \rangle e_k. \]
\end{definition}

\begin{theorem}\label{th:fourier}
    Для любого вектора $x\in H$ ряд Фурье сходится, причем сходится к вектору $x$.
\end{theorem}

\begin{proof}
    По лемме \ref{le:series} ряд Фурье сходится в точности тогда, когда сходится ряд
    $\sum\limits_{k=1}^\infty \absv{\langle x, e_k \rangle}^2$. По неравенству Бесселя (см. <<Лекции по алгебре>>, параграф 17)
    \[ \norm{\sum_{k=1}^n \langle x, e_k \rangle e_k}^2 = \sum_{k=1}^n \absv{\langle x, e_k \rangle}^2 \leq \norm{x}^2, \]
    откуда получаем, что ряд Фурье сходится (последовательность частичных сумм ограничена).
    Обозначим через $y$ сумму этого ряда.

    Покажем, что $x = y$:
    \begin{multline*}
         \langle x - y, e_j \rangle = 
            \langle x - \sum_{k=1}^\infty \langle x, e_k \rangle e_k, e_j \rangle = \\ =
            \langle x, e_j \rangle - \sum_{k=1}^\infty \langle x, e_k \rangle 
                \langle e_k, e_j \rangle =
                \langle x, e_j \rangle - \langle x, e_j \rangle = 0.
    \end{multline*}

    В силу свойства полноты базиса, $x - y = 0$. Дальнейшее очевидно.
\end{proof}

\begin{corollaryth}[равенство Парсеваля]
    Для любого вектора $x\in H$
    \[ \norm{x}^2 = \sum_{k=1}^\infty \absv{\langle x, e_k \rangle}^2. \]
\end{corollaryth}

\begin{theorem}\label{th:basis}
    Для всякого бесконечномерного сепарабельного гильбертова пространства существует 
    ортонормированный базис Шаудера.
\end{theorem}

\begin{proof}
    Пусть $\menge{y_n} \subset H$ --- счетное всюду плотное множество. Применяя процесс Грама-Шмидта (см. алгебру), получим не более чем счетное 
    ортонормированное множество $M = \menge{e_n}$. Линейная оболочка $\linspan M$,
    как нетрудно видеть, плотна в $H$ (в силу процесса Грама-Шмидта, всякий вектор $y_n$ 
    выражается как конечная линейная комбинация векторов из $M$). 

    Покажем, что $M$ обладает свойством полноты. Пусть 
    \begin{equation}\label{eq:eq1}
        \langle a, e_k \rangle = 0, \quad k \in \mathbb N.
    \end{equation}
    Рассмотрим последовательность подпространств
    \[ E_n = \linspan \menge{e_1, \dotsc, e_n}. \] 
    В силу условия $\overline{\linspan M} = H$,
    \begin{equation}\label{eq:eq2}
        d(a, E_n) \to 0,
    \end{equation}    
    где $d(a, E_n) = \inf\limits_{z \in E_n} \norm{a - z}$.

    По теореме \ref{th:projection}, проекция 
    $a_n = \sum\limits_{k=1}^n \langle a, e_k \rangle e_k$ 
    вектора $a$ на $E_n$ есть элемент наилучшего приближения, то есть
    \[ d(a, E_n) = \norm{a - a_n}. \]
    Но $a_n = 0$ для всех $n$ в силу условия \eqref{eq:eq1}. Поэтому
    \[ d(a, E_n) = \norm{a}, \]
    откуда получаем, что $a = 0$ в силу \eqref{eq:eq2}.

    Таким образом, $\menge{e_n}$ --- базис в $H$.
\end{proof}

\begin{definition}
    Два нормированных пространства $X$ и $Y$ называют \emph{изометрически изоморфными}, если
    существует такой биективный оператор $J \in L(X, Y)$, что для всех $x \in X$
    \[ \norm{Jx} = \norm{x}. \]
\end{definition}

\begin{theorem}
    Каждое сепарабельное гильбертово пространство бесконечной размерности над полем $\fieldc$
    изометрически изоморфно пространству последовательностей $l^2 = l^2(\mathbb N, \fieldc)$.
\end{theorem}

\begin{proof}
    Рассмотрим произвольный ортонормированный базис $\menge{e_n}$ в $H$, существующий в силу теоремы \ref{th:basis}. 

    Определим оператор $J \colon H \to l^2$ по правилу
    \[ Jx = (\langle x, e_n \rangle)_{n=1}^\infty, \]
    то есть $J$ ставит $x$ в соответствие последовательность его 
    координат $\langle x, e_n \rangle$.

    Инъективность следует из теоремы \ref{th:fourier}, сюръективность из леммы \ref{le:series}.
    Изометричность следует из равенства Парсеваля.
\end{proof}

\begin{corollaryth}
    \indent Все сепарабельные гильбертовы пространства изометрически изоморфны между собой.
\end{corollaryth}

\subsection[Теорема Рисса]{Теорема Рисса об общем виде линейного функционала}
\begin{theorem}[Рисса о представлении]
    Каждый линейный ограниченный функционал $f \in H^*$ допускает единственное 
    представление вида
    \begin{equation}\label{eq:riesz}
        f(x) = \langle x, a \rangle,
    \end{equation}
    где $a \in H$, причем
    \[ \norm{f} = \norm{a}. \]
\end{theorem}

\begin{proofbreak}
\begin{enumerate}
    \item Если $a \in H$ --- фиксированный вектор, то \eqref{eq:riesz}, очевидно, задаёт линейный
        функционал. Определим его норму:
            \[ \absv{f(x)} = \absv{\langle x, a \rangle} \leq \norm{a}\norm{x} \quad x \in H; \]
            \[ \norm{f} \leq \norm{a}; \]
            \[ \absv{f(\frac{a}{\norm{a}})} = \langle a, \frac{a}{\norm{a}} \rangle = \norm{a}. \]
        Таким образом
        \[ \norm{f} = \norm{a} \]
    \item Пусть $f \in H^*$. Будем считать, что $f \neq 0$, потому что в противном случае
        достаточно взять $a = 0$. Тогда $M = \ker f \neq H$.

        Возьмем ненулевой вектор $b \in \ker f$. Очевидно, что (проверяется непосредственно)
        \[ f(x)b - f(b)x \in \ker f. \]
        Тогда $f(x)b - f(b)x \perp b$. В таком случае
        \[ \langle f(x)b - f(b) x, b \rangle = f(x) \langle b, b \rangle - 
            f(b) \langle x, b \rangle = 0, \]
        откуда получаем
        \[ f(x) = \frac{f(b)\langle x, b \rangle}{\norm{b}^2} 
            = \langle x, \frac{\overline{f(b)}}{\norm{b}^2}b \rangle = \langle x, a \rangle, \]
        где $a = \displaystyle\frac{\overline{f(b)}}{\norm{b}^2}b$.
\end{enumerate}
\end{proofbreak}
