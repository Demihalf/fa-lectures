\section{Элементы теории меры и интеграла}
\subsection{Пространства с мерой}

\begin{definition}
    Пусть $X$ --- непустое множество. Семейство подмножеств $\mathcal{F}$ из~$X$ называется
    \emph{$\sigma$-алгеброй}, если выполняются следующие условия:
    \begin{enumerate}
        \item $X \in \mathcal F$;
        \item $X \setminus A \in \mathcal F$ для всех $A$ из $\mathcal F$;
        \item для всех $A_i, \, i
            \in \mathbb N$ из $\mathcal F$ 
            \[\bigcup\limits_{i=1}^\infty A_i \in \mathcal F.\]
    \end{enumerate}

    Подмножества, принадлежащие этому семейству, называются \emph{измеримыми}.
\end{definition}

\begin{definition}
    Отображение $\mu \colon \mathcal F \to \fieldr \cup \menge{\infty}$
    называется \emph{мерой}, если
    \begin{enumerate}
        \item $\mu(A) \geq 0$ для всех измеримых подмножеств $A$;
        \item для любой последовательности
            $\menge{A_i}$ \emph{взаимно непересекающихся} измеримых подмножеств
            справедливо
            \[ \mu\left(\bigcup\limits_{i=1}^\infty A_i\right) =
            \sum\limits_{i=1}^\infty \mu(A_i). \]
    \end{enumerate}
\end{definition}

\begin{theorem}
    Справедливы следующие свойства:
    \begin{enumerate}
        \item Пересечение конечного или счетного числа измеримых
            множеств есть измеримое множество;
        \item Если $E_1$ и $E_2$ --- измеримые множества и $E_1 \subset E_2$,
            то 
            \[ \mu(E_1) \leq \mu(E_2). \]
    \end{enumerate}
\end{theorem}

\begin{proof}
    См. методичку
\end{proof}

\begin{definition}
    Тройка $(X, \mathcal F, \mu)$, где $X$ --- непустое множество, $\mathcal F$
    --- $\sigma$-алгебра измеримых подмножеств из $X$, а $\mu$ --- мера,
    называется \emph{пространством с мерой}.
\end{definition}

\begin{example}
    Пусть~$X$ --- некоторое непустое множество. В~качестве~$\mathcal F$
    возьмем всевозможные подмножества из~$X$. Очевидно, что они образуют
    $\sigma$-алгебру. Меру~$\mu_a \colon \mathcal F \to \mathbb R$, где 
    $a$ --- некоторый элемент из~$X$, определим следующим образом:
    \[ \mu_a(A) = 
        \begin{cases}
            1, & \text{если } a \in A \\
            0 & \text{в противном случае.}
    \end{cases} \]

    Доказательство того, что определенная таким образом функция в самом деле
    является мерой, элементарно (см. методичку).

    Построенная мера называется \emph{мерой Дирака, сосредоточенной в точке~$a$}.
\end{example}

\begin{example}
    В~качестве~$X$ возьмем вещественную прямую $\mathbb R$. Определим длину
    интервала~$(a, b)$ равенством $\mu( (a, b) ) = b - a$. Любое открытое
    множество на~прямой представимо в~виде объединения не~более чем счетного
    числа взаимно непересекающихся интервалов. Тогда определим меру
    открытого множеств по~формуле \[\mu(G) = \sum\limits_{i=1}^\infty (b_i -
    a_i),\] где 
    \[ G = \bigcup\limits_{i=1}^\infty (a_i, b_i). \]

    Пусть~$E \subset \mathbb R$ --- ограниченное множество на~прямой. Его можно
    покрыть некоторым открытым множеством $G \supset E$. Величина 
    \[ \mu^*(E) = \inf\limits_{G \supset E} \mu(G),\] 
    где инфимум берется по~всем открытым покрытиям
    $E$, называется \emph{верхней мерой} множества~$E$.

    \emph{Нижняя мера} множества~$E$ определяется по~формуле 
    \[ \mu_*(E) = b - a - \mu([a, b] \setminus E),\]
    где $[a, b]$ --- наименьший отрезок, содержащий множество $E$.

    Назовём ограниченное множество $E$ \emph{измеримым по Лебегу}, если 
    \[ \mu_*(E) = \mu^*(E).\]
    Тогда \emph{мерой Лебега} множества $E$ назовём общее значение
    верхней и нижней мер этого множества.

    Мера Лебега также определяется и для~неограниченных множеств. Для этого в~качестве
    нижней меры множества $E$ берется предел нижних мер множеств вида
    $E_n = E \cap [-n, n]$ при $n \to \infty$. Этот предел существует или бесконечен,
    поскольку последовательность $\mu_*(E_n)$, как можно показать, монотонно неубывает.
\end{example}

\begin{theorem}
    Тройка $(\mathbb R, \mathcal F, \mu)$, где $\mathcal F$ --- множество
    измеримых по Лебегу множеств на прямой, а $\mu$ --- мера Лебега, является
    пространством с мерой.
\end{theorem}

\begin{example}
    Тройка $(\Omega, \mathfrak A, P)$, где $\Omega$ ---
    пространство элементарных исходов, 
    $\mathfrak A$ --- алгебра событий, $P$ --- вероятностная мера,
    является пространством с мерой.
\end{example}

\subsection{Интегрирование простых функций}

Пусть далее $(X, \mathcal F, \mu)$ --- пространство с мерой, 
$E \in \mathcal F$ --- некоторое измеримое подмножество.

\begin{definition}
    Функция $f \colon E \to \mathbb R$ называется \emph{простой}, если $E$ можно
    представить в виде счетного объединения взаимно непересекающихся измеримых
    подмножеств $E_i$ так, что функция $f$ принимает на этих подмножествах
    постоянное значение: $f(x) = a_i$ для всех $x$ из $E_i$.

    Функция $f$ называется \emph{ступенчатой}, если такое объединение конечно.
\end{definition}

\begin{example}
    Пусть $(\mathbb R, \mathcal F, \mu)$ --- прямая с мерой Лебега, $E = [0,
    1]$. Функция Дирихле, определенная на $E$ и принимающая значение $1$ 
    для рациональных аргументов и $0$ для иррациональных, является простой (и
    даже ступенчатой).
    В качестве $E_1$ можно взять множество рациональных чисел из отрезка $E$, а
    в качестве $E_2$ --- множество иррациональных чисел из того же отрезка. Оба
    этих множества измеримы по Лебегу.
\end{example}

\begin{lemma}\label{le:lincombsimple}
    Линейная комбинация простых функций, определенных на измеримом множестве $E$
    является простой функцией.
\end{lemma}

\begin{proof}
    Покажем, что $\alpha f + \beta g$ также простая функция для простых функций
    $f, g \colon E \to \fieldr$ и чисел $\alpha, \beta \in \fieldr$.

    Пусть
    \[ E = \bigcup_{i=1}^\infty E_i = \bigcup_{j=1}^\infty F_j, \]
    причем 
    \[ f(x) = a_i, \quad x \in E_i, \]
    \[ g(x) = b_j, \quad x \in F_j. \]

    Обозначим $G_{ij} = E_i \cap F_j$. Это также измеримые множества. Более того
    непосредственно проверяется, что
    \[ E = \bigcup_{i, j = 1}^\infty G_{ij}. \]

    На множестве $G_{ij}$ функция $\alpha f + \beta g$ принимает значение
    \[ (\alpha f + \beta g) = \alpha a_i + \beta b_j. \]

    Этим доказано, что функция $\alpha f + \beta g$ простая, принимающая
    постоянные значения на множествах $G_{ij}$.
\end{proof}

Из этой леммы следует, что простые функции образуют линейное пространство.

Далее будем считать, что мера множества $E$ конечна.

\begin{definition}
    Простая функция $f \colon E \to \mathbb R$ называется \emph{абсолютно
    суммируемой,}
    если конечна величина 
    \[ \sum_{i=1}^\infty \absv{a_i} \mu(E_i), \]
    в обозначениях предыдущего определения.
\end{definition}

\begin{definition}
    \emph{Интегралом} от абсолютно суммируемой функции $f$ называется сумма вида
    \[ \int_E f(x) \dd \mu(x) := \sum_{i=1}^\infty a_i \mu(E_i). \]
\end{definition}

Аргумент в записи интеграла часто опускают и пишут просто
\[ \int_E f \dd \mu. \]

В следующей теореме доказываются основные свойства интеграла от абсолютно
суммируемых функций.
\begin{theorem}
    Пусть $f, g \colon E \to \mathbb R$ --- абсолютно суммируемые функции.
    Тогда справедливы следующие свойства:
    \begin{enumerate}
        \item Линейность: для любых $\alpha, \beta \in \fieldr$ функция 
            $\alpha f + \beta g$ абсолютно суммируема и справедливо равенство
            \[ \int_E (\alpha f + \beta g) \dd \mu = \alpha \int_E f \dd \mu +
            \beta \int_E g \dd \mu; \]
        
        \item Оценка модуля интеграла:
            \[ \absv{\int_E f \dd \mu} \leq  \mu(E) \sup_{x \in E} \absv{f(x)};
            \]

        \item Неотрицательность: если $f \geq 0$, то 
            \[ \int_E f \dd \mu \geq 0; \]

        \item Монотонность: если $f \geq g$, то
            \[ \int_E f \dd \mu \geq \int_E g \dd \mu; \]

        \item Аддитивность: если $E$ представимо в виде объединения не более чем
            счетного числа взаимно непересекающихся измеримых подмножеств $A_k$,
            то
            \[ \int_E f \dd \mu = \sum_k \int_{A_k} f \dd \mu. \]

    \end{enumerate}
\end{theorem}

\begin{proofbreak}
    \begin{enumerate}
        \item Абсолютная суммируемость линейной комбинации следует из
            леммы \ref{le:lincombsimple}, свойств абсолютно сходящихся числовых рядов и из свойства
            монотонности меры.

            Покажем, что справедливо указанное в утверждении теоремы равенство.
            Будем пользоваться обозначениями из леммы.
            \begin{multline*}
                \int_E (\alpha f + \beta g) \dd \mu = \sum_{i,j=1}^\infty (\alpha
                a_i + \beta b_j) \mu (G_{ij}) = \\ = \alpha \sum_{i=1}^\infty
                \sum_{j=1}^\infty a_i \mu (G_{ij}) + \beta \sum_{i=1}^\infty
                \sum_{j=1}^\infty b_j \mu (G_{ij}) = \\ = \alpha \sum_{i=1}^\infty
                a_i \sum_{j=1}^\infty \mu (G_{ij}) + \beta \sum_{j=1}^\infty b_j
                \sum_{i=1}^\infty \mu (G_{ij}).
            \end{multline*}

            Поскольку, как нетрудно видеть, 
            \[ E_i = \bigcup\limits_{j=1}^\infty G_{ij}, \quad 
                F_j = \bigcup\limits_{i=1}^\infty G_{ij},\]
            а множества
            $G_{ij}$ взаимно не пересекаются, из свойства аддитивности меры
            получаем
            \[ \sum_{j=1}^\infty \mu (G_{ij}) = \mu(E_i); \quad
             \sum_{i=1}^\infty \mu (G_{ij}) = \mu(F_j). \]

             Таким образом
             \begin{multline*}
                \alpha \sum_{i=1}^\infty
                a_i \sum_{j=1}^\infty \mu (G_{ij}) + \beta \sum_{j=1}^\infty b_j
                \sum_{i=1}^\infty \mu (G_{ij})=   
                \alpha \sum_{i=1}^\infty
                a_i \mu(E_i) +\\+ \beta \sum_{j=1}^\infty b_j
                 \mu(F_j) = \alpha \int_E f \dd \mu + \beta \int_E g \dd \mu. 
             \end{multline*}

        \item Тривиально (неравенство треугольника, аддитивность меры).
        \item Тривиально.
        \item Рассмотреть функцию $f - g$ и применить линейность и предыдущее
            свойство.
        \item Рассмотреть взаимно непересекающиеся множества вида $H_{ik} = E_i \hm \cap A_k$, 
            на которых функция принимает постоянные значения $c_{ik}$, и которые образуют
            разбиение $E$:
            \[ \int_E f \dd \mu = \sum_i \sum_k c_{ik} \mu(H_{ik}) = \sum_k
            \sum_i c_{ik} \mu(H_{ik}) = \sum_k \int_{A_k} f \dd \mu \]
    \end{enumerate}
\end{proofbreak}

\subsection{Интегрирование измеримых функций}
\begin{definition}
    Функция $f \colon E \to \fieldr$, определенная на измеримом множестве $E$,
    называется \emph{измеримой}, если она является равномерным пределом на $E$
    последовательности простых функций, т.е. существует такая последовательность
    $\menge{f_n},\, f_n \colon E \to \fieldr$, что
    \[ \sup_{x\in E} \absv{f(x) - f_n(x)} \to \infty, \quad n \to \infty. \]
\end{definition}

\begin{definition}
    Функция $f \colon E \to \fieldr$ называется \emph{измеримой}, если
    \[ f^{-1}( (-\infty, x) ) \in \mathcal F, \quad \forall x \in \fieldr. \]
\end{definition}

\begin{theorem}
    Вышеприведенные определения измеримой функции эквивалентны.
\end{theorem}

\begin{proof}
    см. в методичке на с. 51 (требуется только необходимость).
\end{proof}

\begin{definition}
    Если существует последовательность простых интегрируемых функций,
    сходящаяся равномерно к измеримой функции $f$, то \emph{интегралом} функции $f$
    назовем предел
    \[ \int_E f \dd \mu := \lim_{n\to \infty} \int_E f_n \dd \mu. \]
\end{definition}

Можно показать, что предел (быть может, бесконечный) всегда существует и не
зависит от выбора последовательности $f_n$.

\begin{definition}
    Неотрицательная функция $f$ называется \emph{интегрируемой} на множестве $E$, если предел из
    предыдущего определения конечен.
\end{definition}

Всякая измеримая функция $f$ представима в виде разности двух неотрицательных
измеримых функций:
\[ f_+(x) = 
    \begin{cases}
        f(x), & f(x) \geq 0 \\
        0, & f(x) < 0
    \end{cases}, \quad
   f_-(x) = 
    \begin{cases}
        -f(x), & f(x) \leq 0 \\
        0, & f(x) > 0
\end{cases}. \]
\[ f(x) = f_+(x) - f_-(x). \]

Тогда если хотя бы одна из функций $f_+$ или $f_-$ интегрируема, интегралом
функции $f$ назовём величину
\[ \int_E f \dd \mu = \int_E f_+ \dd \mu - \int_E f_- \dd \mu. \]

\begin{definition}
    В случае, когда $X = \fieldr$, $\mathcal F$ --- $\sigma$-алгебра измеримых по Лебегу
    множеств на $\fieldr$, $\mu$ --- мера Лебега, интеграл, определённый по схеме,
    приведённой в данном разделе, называется \emph{интегралом Лебега} на прямой.
\end{definition}

\begin{theorem}
    Если $(\fieldr, \mathcal F, \mu)$ --- прямая с мерой Лебега, $f \colon [a,
    b] \to \fieldr$ интегрируема по Риману, то тогда она интегрируема по Лебегу
    и значения интегралов Римана и Лебега совпадают.
\end{theorem}

\subsection{Пространства Лебега}
\begin{definition}
    Функция $f \colon E \to \fieldr$, определенная на измеримом множестве $E$,
    называется \emph{суммируемой со степенью $p$}, $p \geq 1$, если величина
    \[ \int_{E} \absv{f(x)}^p \dd \mu(x) \]
    определена и конечна.
\end{definition}

\begin{definition}
    Будем говорить, что некоторое свойство выполнено \emph{почти всюду} на измеримом
    множестве $E$, если оно выполнено на всём множестве $E$, за исключением,
    быть может, множества меры нуль.
\end{definition}

\begin{definition}
    Две функции $f_1, f_2 \colon E \to \fieldr$ назовём \emph{эквивалентными} на множестве $E$, если их
    значения совпадают почти всюду.
\end{definition}

Отношение $\sim$, введённое в определении выше, является отношением эквивалентности.

Пусть $\mathcal L^p(E, \mu), \, p \geq 1$ --- линейное пространство суммируемых со степенью
$p$ функций, определенных на множестве $E$.

Рассмотрим фактормножество $L^p(E, \mu) = \mathcal L^p(E, \mu)/\!\sim$. Оно также будет являться линейным
пространством. В нём можно ввести норму по формуле
\[ \|\tilde{f}\|_p = \left( \int_E \absv{f(x)}^p \dd \mu(x)
\right)^{1/p}. \]

Классы эквивалентности из $L^p(E, \mu)$, допуская неточность, часто отождествляют с
функ\-ци\-я\-ми-представителями из этого класса.

Если $E = [a,b] \subset \fieldr$, $\mu$ --- мера Лебега на прямой, то вместо
$L^p([a,b], \mu)$ обычно пишут просто $L^p[a,b]$.

\begin{theorem}[Лебега]
    $L^p(E, \mu)$ --- банахово пространство.
\end{theorem}
