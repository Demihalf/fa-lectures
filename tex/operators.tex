\section{Ограниченные операторы}

Далее $X$ и $Y$ --- нормированные пространства над полем $\fieldk \in \menge{\fieldr,
\fieldc}$.

\begin{definition}
    Отображение $A \colon X \to Y$ называется \emph{линейным оператором}, действующим
    из пространства $X$ в $Y$, если
    \[ A(\alpha x_1 + \beta x_2) = \alpha Ax_1 + \beta Ax_2, \quad \forall x_1,
    x_2 \in X, \alpha, \beta \in \fieldk. \]
\end{definition}

Если $Y = \fieldk$, то вместо слова <<оператор>> говорят <<функционал>>.

\begin{example}
    Отображение $D \colon C^1[a, b] \to C[a, b]$, определённое по правилу
    $Dx = x'$ называется \emph{оператором дифференцирования}. Это линейный
    оператор.
\end{example}

\begin{example}
    Отображение $J \colon C[a, b] \to C[a, b]$, определённое по правилу
    \[ (Jx)(t) = \int_{a}^t x(s) \dd s, \quad t \in [a, b], \]
    назывется \emph{оператором неопределённого интегрирования.}
\end{example}

\begin{example}
    Пусть $(\Omega, \mathcal F, \mu)$ --- пространство с мерой, $L^1(\Omega,
    \mu)$ --- банахово пространство классов эквивалентности суммируемых функций
    на $\Omega$. Отображение $J_0 \colon L^1(\Omega, \mu) \to \fieldr$,
    определенное по правилу
    \[ J_0 x = \int_\Omega x \dd \mu, \]
    есть линейный функционал.
\end{example}

\begin{example}
    Отображение $A \colon \ell^1 \to \ell^\infty$, определённое по правилу
    \[ (Ax)(n) = \sum_{k=1}^n x(k), \]
    есть линейный оператор, который каждой последовательности из $\ell^1$ ставит
    в соответствие её последовательность частичных сумм.
\end{example}

\begin{example}
    Отображение $A \colon C[a, b] \to C[a, b]$, определённое по правилу
    \[ (Ax)(t) = \int_a^b K(t, s) x(s) \dd s, \quad t \in [a,b], \]
    где $K : [a, b] \times [a, b] \to \fieldr$ --- непрерывная функция,
    называется \emph{интегральным оператором}. При этом функция $K$ называется
    \emph{ядром}
    этого интегрального оператора.
\end{example}

\begin{definition}
    Оператор $A \colon X \to Y$ между нормированными пространствами 
    называется \emph{ограниченным}, если величина
    \[ \norm{A} = \sup_{\norm{x} \leq 1} \norm{Ax} \]
    конечна. Эта величина, в таком случае, называется нормой оператора $A$.
\end{definition}

Можно показать, что все следующие определения нормы совпадают с данным выше:
\begin{enumerate}
    \item $ \norm{A} = \sup\limits_{\norm{x} < 1} \norm{Ax} $
    \item $ \norm{A} = \sup\limits_{\norm{x} = 1} \norm{Ax} $
    \item $ \norm{A} = \sup\limits_{x\neq 0}\displaystyle\frac{\norm{Ax}}{\norm{x}}; $
    \item $ \norm{A} = \inf\menge{C \geq 0 : \forall x \in X \; \norm{Ax} \leq C
        \norm{x}} $
\end{enumerate}

Нетрудно видеть, что $\norm{Ax} \leq \norm{A}\norm{x}$ для всех $x \in X$.

\begin{example}
    Рассмотрим оператор умножения $A \colon \fieldc \to \fieldc$, $\\Ax = ax$,
    где $a \in \fieldc$. Если $\norm{x} = \absv{x} = 1$, то 
    \[ \norm{Ax} = \absv{ax} = \absv{a}. \]
    Таким образом $\norm{A} = \absv{a}$.
\end{example}

Множество всех линейных ограниченных операторов между \\ нормированными
пространствами $X$ и $Y$ будем обозначать $L(X, Y)$.

\begin{theorem}
    $L(X, Y)$ --- нормированное пространство.
\end{theorem}

\begin{proof}
    Непосредственно доказывается, что сумма ограниченных
    операторов есть ограниченный оператор. Также легко показать, что норма
    оператора --- в самом деле норма в $L(X, Y)$. Установим, например,
    справедливость неравенства треугольника. Пусть $A, B \in L(X, Y)$ и
    $\norm{x} = 1$. Тогда
    \[ \norm{(A+B)x} = \norm{Ax+Bx} \leq \norm{Ax} + \norm{Bx}. \]
    Взяв верхнюю грань по всем $x$ с нормой $1$, получим, что 
    \[ \norm{A + B} \leq \norm{A} + \norm{B}. \]
\end{proof}

\begin{theorem}
    Если $Y$ --- банахово пространство, то $L(X, Y)$ --- банахово
    пространство.
\end{theorem}

\begin{proof}
   см. Антоневич, Радыно, 1984, с. 180. 
\end{proof}

\begin{definition}
    Если $f \colon X \to \fieldk$ и $f$ --- линейный оператор,
     то $f$ называют \emph{линейным функционалом} на $X$.

    Пространство ограниченных линейных функционалов $L(X, \fieldk)$ называют
    \emph{сопряженным пространством} к пространству $X$ и обозначают символом $X^*$.
\end{definition}

\begin{theorem}
    Пусть $A \in L(X, Y)$. Тогда следующие условия эквивалентны:
    \begin{enumerate}
        \item $A$ --- непрерывное отображение;
        \item $A$ --- непрерывное в точке $0$ отображение;
        \item $A$ --- ограниченный оператор;
        \item $A$ --- липшицево отображение.
    \end{enumerate}
\end{theorem}

\begin{proof}
    Импликации $1 \Rightarrow 2$, и $4 \Rightarrow 1$ очевидны.
    Докажем, что $2 \Rightarrow 3$.
    Непрерывность $A$ означает, что
    \[ \forall \varepsilon > 0 \; \exists \delta > 0 \; \forall x \in X: \; \norm{x} <
    \delta \rightarrow \norm{Ax} < \varepsilon. \]
    Зафиксируем некоторый $\varepsilon > 0$ и соответствующий ему $\delta$.
    Тогда для любого $x \in X$, $\norm{x} \leq
    1$, справедливо
    \[ \norm{Ax} = \frac{2}{\delta} \norm{A\left( \frac{\delta}{2} x \right)}
    \leq \frac{2\varepsilon}{\delta}. \]
    Переходя в неравенстве к верхней грани, получаем, что
    \[ \sup_{\norm{x} \leq 1}\norm{Ax} \leq \frac{2\varepsilon}{\delta}, \]
    что и означает ограниченность оператора $A$.

    Импликация $3 \Rightarrow 4$ проверяется непосредственно: если $A$ ---
    ограниченный оператор, $x_1, x_2 \in X$, то
    \[ \norm{Ax_1 - Ax_2} = \norm{A(x_1 - x_2)} \leq \norm{A} \norm{x_1 - x_2}.
    \]
\end{proof}
