\section{Ряды в банаховом пространстве}
\begin{definition}
    \emph{Рядом} элементов из нормированного пространства $X$ называется пара
    последовательностей $(x_n, s_n)$, связанных соотношением
    \[ s_n = \sum_{k=1}^n x_k. \]
    $x_n$ называют \emph{$n$-ым членом} ряда, а $s_n$ --- \emph{$n$-ой частичной
    суммой} ряда.
\end{definition}

\begin{definition}
    Говорят, что ряд $(x_n, s_n)$ сходится, если сходится последовательность его
    частичных сумм. Тогда предел этой последовательности называют суммой ряда и
    обозначают
    \[ \lim_{n\to\infty} s_n = \sum_{k=1}^\infty x_k. \]
\end{definition}

\begin{definition}
    Говорят, что ряд $(x_n, s_n)$ \emph{абсолютно сходится}, если сходится числовой ряд
    вида
    \[ \sum_{k=1}^\infty \norm{x_k}. \]
\end{definition}

\begin{theorem}
    Если ряд элементов из банахова пространства 
    сходится абсолютно, то он сходится.
\end{theorem}

\begin{theorem}
    Пусть задан ряд $(x_n, s_n)$ элементов из банахова пространства $X$ и
    существует числовой ряд $a_n$ такой, что для всех $n$ выполняется
    неравенство
    \[ \norm{x_n} \leq a_n. \]
    Тогда ряд $(x_n, s_n)$ сходится абсолютно.
\end{theorem}

Эти теоремы доказываются аналогично знакомым теоремам из курса математического
анализа.
